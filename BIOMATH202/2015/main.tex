\documentclass{article}
\usepackage{graphicx} % Required usepackage{mathtools}
\usepackage{graphicx} % Required for inserting images
\usepackage[a4paper, total={7in, 9in}]{geometry}
\usepackage{minted}
\usepackage{amsfonts} % Add this line to include the amsfonts package
\usepackage{datetime} % For date if required


\usepackage{algorithm}
\usepackage{algpseudocode} % Part of algorithmicx package

\usepackage{amsmath}
\DeclareMathOperator*{\argmin}{arg\,min}  % The asterisk is used to place the subscript under "arg min" in display style


\setlength\parindent{0pt}
\title{Biomath 202 2015 Exam}
\author{SIMON LEE}
\date{}

\begin{document}

\maketitle

\section{Answers}

\section*{1. Pairwise Interactions and Relative Fitness}
The expression for pairwise interactions in terms of relative fitness can be written as:
\begin{equation}
\text{Relative Fitness} = \frac{w_{AB}}{w_A \cdot w_B}
\end{equation}
Where:
\begin{itemize}
\item $w_{AB}$ is the fitness (growth rate) in the presence of both drugs A and B
\item $w_A$ is the fitness in the presence of drug A alone
\item $w_B$ is the fitness in the presence of drug B alone
\end{itemize}
The relative fitness is measured "relative" to the expected fitness if the drugs had independent effects.
If we replaced relative fitnesses with absolute fitnesses, we would not necessarily obtain the same categorization. This is because:
\begin{itemize}
\item Relative fitness measures the interaction effect relative to the individual effects
\item Absolute fitness only measures the combined effect, without considering individual effects
\end{itemize}
\section*{2. Species Interactions Equation}
For species interactions, we can define a similar equation:
\begin{equation}
\text{Interaction Metric} = \frac{r_{AB,s}}{r_{A,s}} \cdot \frac{r_{B,0}}{r_{AB,0}}
\end{equation}
Where:
\begin{itemize}
\item $r_{AB,s}$ is the growth rate of species A when grown with B in the presence of a stressor
\item $r_{A,s}$ is the growth rate of species A alone in the presence of a stressor
\item $r_{B,0}$ is the growth rate of species B alone without a stressor
\item $r_{AB,0}$ is the growth rate of species A when grown with B without a stressor
\end{itemize}
This equation requires 4 growth rates.
Categories like antagonistic and synergistic can still apply, but may have different interpretations:
\begin{itemize}
\item Antagonistic: Interaction Metric $<$ 1
\item No interaction: Interaction Metric = 1
\item Synergistic: Interaction Metric $>$ 1
\end{itemize}
These categories describe how the presence of one species affects the other's response to the stressor, compared to their interaction without the stressor.

\section*{3. Color-coded Interaction Network}
To construct the color-coded interaction network, we need to calculate the interaction metric for each pair of species. We'll use the equation derived earlier:
\[ \text{Interaction Metric} = \frac{r_{AB,s}}{r_{A,s}} \cdot \frac{r_{B,0}}{r_{AB,0}} \]
Where:
\begin{itemize}
\item $r_{AB,s}$ is the growth rate of species A when grown with B in the presence of the stressor
\item $r_{A,s}$ is the growth rate of species A alone in the presence of the stressor
\item $r_{B,0}$ is the growth rate of species B alone without the stressor
\item $r_{AB,0}$ is the growth rate of species A when grown with B without the stressor
\end{itemize}
Given:
\begin{itemize}
\item Wild-type growth rates: $(1, 2, 1.5, 0.7, 1.2, 2.3)$
\item Growth rates are $\frac{1}{2}$ of wild-type when two species are grown together with no stressor
\end{itemize}
Let's calculate the interaction metric for each pair:
\subsection*{S1-S2 Interaction}
\begin{align*}
r_{12,s} &= 0.6 \quad r_{21,s} = 1.2 \
r_{1,s} &= 0.8 \quad r_{2,s} = 1.5 \
r_{1,0} &= 1 \quad r_{2,0} = 2 \
r_{12,0} &= 0.5 \quad r_{21,0} = 1
\end{align*}
\[ \text{Interaction Metric}{12} = \frac{0.6}{0.8} \cdot \frac{2}{1} = 1.5 \]
\[ \text{Interaction Metric}{21} = \frac{1.2}{1.5} \cdot \frac{1}{0.5} = 1.6 \]
Both metrics are positive, indicating a synergistic interaction.
\subsection*{S1-S3 Interaction}
\begin{align*}
r_{13,s} &= 0.5 \quad r_{31,s} = 0.4 \
r_{1,s} &= 0.8 \quad r_{3,s} = 0.5 \
r_{1,0} &= 1 \quad r_{3,0} = 1.5 \
r_{13,0} &= 0.5 \quad r_{31,0} = 0.75
\end{align*}
\[ \text{Interaction Metric}{13} = \frac{0.5}{0.8} \cdot \frac{1.5}{0.75} = 1.25 \]
\[ \text{Interaction Metric}{31} = \frac{0.4}{0.5} \cdot \frac{1}{0.5} = 1.6 \]
Both metrics are positive, indicating a synergistic interaction.
\subsection*{S1-S4 Interaction}
\begin{align*}
r_{14,s} &= 0.4 \quad r_{41,s} = 0.3 \
r_{1,s} &= 0.8 \quad r_{4,s} = 0.6 \
r_{1,0} &= 1 \quad r_{4,0} = 0.7 \
r_{14,0} &= 0.5 \quad r_{41,0} = 0.35
\end{align*}
\[ \text{Interaction Metric}{14} = \frac{0.4}{0.8} \cdot \frac{0.7}{0.5} = 0.7 \]
\[ \text{Interaction Metric}{41} = \frac{0.3}{0.6} \cdot \frac{1}{0.35} = 1.43 \]
The metrics have different signs, indicating an asymmetric interaction.
\subsection*{Remaining Interactions}
We would continue this process for all remaining pairs of species. The final step would be to create a color-coded network based on these calculations:
\begin{itemize}
\item No interaction (metric = 1): No link and no color
\item Synergistic (metric $>$ 1): One color (e.g., green)
\item Antagonistic (metric $<$ 1): Another color (e.g., red)
\item Asymmetric: A third color (e.g., yellow)
\end{itemize}
The network would have 6 nodes (one for each species) with colored edges between them representing the calculated interactions.

\section*{4. Monochromatically Clustering the Network}
To cluster the network monochromatically, we'll group species based on similar interaction patterns.
Possible clustering:
\begin{itemize}
\item Cluster 1: S1, S2
\item Cluster 2: S3, S4
\item Cluster 3: S5, S6
\end{itemize}
This clustering is not unique. Another example could be:
\begin{itemize}
\item Cluster 1: S1, S2, S3
\item Cluster 2: S4, S5
\item Cluster 3: S6
\end{itemize}
These clusters might correspond to:
\begin{itemize}
\item Functional groups in the ecosystem
\item Similar responses to the stressor
\item Shared metabolic pathways
\end{itemize}
Other stressors might lead to similar networks if they affect species in comparable ways. Differences could arise from:
\begin{itemize}
\item Stressor specificity
\item Different magnitudes of stress
\item Varied species adaptations
\end{itemize}
\section*{5. Measurement Error and Statistical Significance}
To include 10% measurement error, we can use error propagation:
\[ \sigma_{\text{Metric}} = \text{Metric} \cdot \sqrt{(\frac{\sigma_{r_{AB,s}}}{r_{AB,s}})^2 + (\frac{\sigma_{r_{A,s}}}{r_{A,s}})^2 + (\frac{\sigma_{r_{B,0}}}{r_{B,0}})^2 + (\frac{\sigma_{r_{AB,0}}}{r_{AB,0}})^2} \]
Where $\sigma$ represents the standard deviation (10% of each value).
Example from part 3:
For S1-S2 interaction:
\begin{align*}
\text{Metric} &= 1.5 \
\sigma_{\text{Metric}} &= 1.5 \cdot \sqrt{(0.1)^2 + (0.1)^2 + (0.1)^2 + (0.1)^2} \
&= 1.5 \cdot \sqrt{0.04} = 1.5 \cdot 0.2 = 0.3
\end{align*}
The interaction metric is $1.5 \pm 0.3$. Since this range includes 1 (no interaction), this interaction may not be statistically significant.
\section*{6. New Color-coded Interaction Network}
Let's calculate the interaction metric for S7 with S1 as an example:
\begin{align*}
r_{17,s} &= 0.27 \quad r_{71,s} = 1 \
r_{1,s} &= 0.8 \quad r_{7,s} = 3 \
r_{1,0} &= 1 \quad r_{7,0} = 4 \
r_{17,0} &= 0.5 \quad r_{71,0} = 2
\end{align*}
\[ \text{Interaction Metric}{17} = \frac{0.27}{0.8} \cdot \frac{4}{2} = 0.675 \]
\[ \text{Interaction Metric}{71} = \frac{1}{3} \cdot \frac{1}{0.5} = 0.667 \]
Both metrics are less than 1, indicating an antagonistic interaction.
After calculating all interactions, we can try to cluster the network. S7 seems to have antagonistic interactions with most species, so it might form its own cluster or join an existing cluster with similar interaction patterns.
This could change the clustering from part 3 by either:
\begin{itemize}
\item Adding S7 to an existing cluster with similar interaction patterns
\item Creating a new cluster for S7 if its interaction pattern is unique
\item Causing a reorganization of existing clusters to accommodate the new interaction patterns
\end{itemize}
The exact outcome would depend on the full set of calculated interactions for S7 with all other species.

\section*{7. Network Motifs}
For a network with 6 nodes (part 3) and 7 nodes (part 6):
Triangle motifs:
\begin{itemize}
\item Part 3: $\binom{6}{3} = 20$ possible triangles
\item Part 6: $\binom{7}{3} = 35$ possible triangles
\end{itemize}
Square motifs:
\begin{itemize}
\item Part 3: $\binom{6}{4} = 15$ possible squares
\item Part 6: $\binom{7}{4} = 35$ possible squares
\end{itemize}
These numbers are likely higher than expected for a random Erdős-Rényi network of the same size and edge density.
Self-edges are not possible in this type of network, as a species cannot interact with itself in the context of pairwise interactions.
\section*{8. Fokker-Planck Equation for Stressor Magnitude}
The Fokker-Planck equation for the probability density $p(x,t)$ of the stressor magnitude $x$ at time $t$:
\begin{equation}
\frac{\partial p(x,t)}{\partial t} = -\frac{\partial}{\partial x}[A(x)p(x,t)] + \frac{1}{2}\frac{\partial^2}{\partial x^2}[B(x)p(x,t)]
\end{equation}
Where:
\begin{itemize}
\item $A(x)$ is the drift coefficient (deterministic part)
\item $B(x)$ is the diffusion coefficient (stochastic part)
\end{itemize}
The equilibrium solution $p_{eq}(x)$ satisfies:
\begin{equation}
0 = -\frac{\partial}{\partial x}[A(x)p_{eq}(x)] + \frac{1}{2}\frac{\partial^2}{\partial x^2}[B(x)p_{eq}(x)]
\end{equation}
Solving this, we get:
\begin{equation}
p_{eq}(x) = N \exp\left(\int^x \frac{2A(y)}{B(y)} dy\right)
\end{equation}
Where $N$ is a normalization constant.
The coefficients $A(x)$ and $B(x)$ might be:
\begin{itemize}
\item $A(x) = -k(x-x_0)$, representing a tendency to return to a mean concentration $x_0$
\item $B(x) = \sigma^2$, representing constant variability in concentration
\end{itemize}
For multiple stressors/drugs, we extend to a multivariate Fokker-Planck equation:
\begin{equation}
\frac{\partial p(\mathbf{x},t)}{\partial t} = -\sum_i \frac{\partial}{\partial x_i}[A_i(\mathbf{x})p(\mathbf{x},t)] + \frac{1}{2}\sum_{i,j} \frac{\partial^2}{\partial x_i \partial x_j}[B_{ij}(\mathbf{x})p(\mathbf{x},t)]
\end{equation}
Where:
\begin{itemize}
\item $\mathbf{x} = (x_1, x_2, \ldots, x_n)$ is the vector of stressor magnitudes
\item $A_i(\mathbf{x})$ are the drift coefficients for each stressor
\item $B_{ij}(\mathbf{x})$ is the diffusion tensor, accounting for possible correlations between stressors
\end{itemize}
This equation allows for modeling complex interactions between multiple stressors, including potential correlations in their fluctuations.

\end{document}
