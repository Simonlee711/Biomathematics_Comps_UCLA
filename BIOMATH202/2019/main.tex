\documentclass{article}
\usepackage{graphicx} % Required usepackage{mathtools}
\usepackage{graphicx} % Required for inserting images
\usepackage[a4paper, total={7in, 9in}]{geometry}
\usepackage{minted}
\usepackage{amsfonts} % Add this line to include the amsfonts package
\usepackage{datetime} % For date if required


\usepackage{algorithm}
\usepackage{algpseudocode} % Part of algorithmicx package

\usepackage{amsmath}
\DeclareMathOperator*{\argmin}{arg\,min}  % The asterisk is used to place the subscript under "arg min" in display style


\setlength\parindent{0pt}

\title{Biomath 202 2019 Exam}
\author{SIMON LEE}
\date{}

\begin{document}

\maketitle

\section{Introduction}

\textbf{a.} The SIR model equations represent the rates of change of three population groups:
\begin{itemize}
    \item - Susceptible (\(S\)): \(\frac{dS}{dt} = -\beta IS\)
    \item - Infected (\(I\)): \(\frac{dI}{dt} = \beta IS - \gamma I\)
    \item - Recovered (\(R\)): \(\frac{dR}{dt} = \gamma I\)
\end{itemize}

The total population \(N = S + I + R\). Adding the equations:
\[\frac{dN}{dt} = \frac{dS}{dt} + \frac{dI}{dt} + \frac{dR}{dt} = -\beta IS + \beta IS - \gamma I + \gamma I = 0\]

So total population is conserved.

The fixed points are found by setting all derivatives to zero. This gives:
1) \(S = N\), \(I = R = 0\) (disease-free equilibrium)  
2) \(S^* = \frac{\gamma}{\beta}\), \(I^* = \mu(\beta N - \gamma)/(\beta\gamma)\), \(R^* = (\beta-\mu)N/\beta\), where \(\mu = \frac{\gamma}{\beta N}\) (endemic equilibrium)

\(S^* = \frac{\gamma}{\beta}\) is the herd immunity threshold, below which the disease will spread.\\

\textbf{b.} To incorporate vaccination at rate \(v\) and GBS deaths at rate \(\delta\):
\[\frac{dS}{dt} = -\beta IS - vS\]
\[\frac{dI}{dt} = \beta IS - \gamma I - \delta I\]  
\[\frac{dR}{dt} = \gamma I + vS\]
\[\frac{dD}{dt} = \delta I\]  (new compartment for GBS deaths)

Now \(\frac{dN}{dt} = -\delta I\), so total population is not conserved due to GBS deaths.

As \(t \rightarrow \infty\), the system will approach an equilibrium with \(I = 0\) and \(S = \frac{\gamma+\delta}{\beta+v}\).\\

\textbf{c.} Let \(p\) be the frequency of the GBS allele, \(q=1-p\) the normal allele frequency. Assuming HWE, the mean fitness is:
\[\bar{w} = p^2 (1-s) + 2pq + q^2\], where \(s=0.1\) is the selection coefficient.  

The variance of \(p\) is \(V(p) = 2pq\), maximized at \(p=0.5\).

From the equation in part b, \(\frac{I^*}{N} = \frac{\delta}{\delta+\gamma}  \Rightarrow  p = \frac{\delta}{\delta+\gamma}\), and \(V(p) = 2p(1-p)\).

The strength of selection against \(p\) is:
\[\frac{\partial \ln(\bar{w})}{\partial p} \approx -2qs = -2(1-p)s\]  

By Wright's equation, the change in \(p\) per generation is:
\[\Delta p = -V(p) \frac{\partial \ln(\bar{w})}{\partial p} = -2pq \cdot -2qs = 4pq^2s\]

Since \(q \approx 1\) for a rare allele, \(\Delta p \approx 4ps\), driven more by the selection coefficient \(s=0.1\) than the variance term \(2pq<<1\). The GBS allele frequency will decrease slowly each generation.\\

\textbf{d.} If the world population is 7 billion and everyone gets vaccinated, then the number of vaccinations needed is simply 7 billion. The number of generations until the GBS mutation is eliminated depends on the selection pressure against it.\\

\textbf{e.} To make stochastic predictions, we can use the Wright-Fisher model. The variance in allele frequency p is $V(p) = p(1-p)/(2N_e)$, where $N_e$ is the effective population size. The selection term from part c is $$\frac{\partial \ln(\bar{w})}{\partial p} \approx -2(1-p)s$$.

Putting these together, the expected change in p per generation is:

$$\mathbb{E}[\Delta p] = -V(p) \frac{\partial \ln(\bar{w})}{\partial p} = -\frac{p(1-p)}{2N_e} \cdot -2(1-p)s = \frac{p(1-p)^2 s}{N_e}$$

At equilibrium, $\mathbb{E}[\Delta p] = 0$, which occurs when $p=0$ (mutation eliminated) or $p=1$ (mutation fixed), assuming $s \neq 0$ and $N_e$ is finite.

The time to elimination depends on the initial p and the population size $N_e$. In a very large population, drift is weak and selection dominates, so elimination would be faster than in a small population where drift is strong relative to selection.

Other factors that could help the GBS allele persist include:
\begin{itemize}
    \item - Balancing selection (heterozygote advantage)
    \item - Linked selection (hitchhiking with a beneficial allele)
    \item - Population structure/migration
    \item - Mutation-selection balance
\end{itemize}

\textbf{f.} If a genetic screen identified carriers, the SIR equations would be:

$$\frac{dS}{dt} = -\beta IS - vS$$
$$\frac{dI}{dt} = \beta IS - \gamma I$$
$$\frac{dR}{dt} = vS + \gamma I$$
$$\frac{dG}{dt} = -vG$$ (screened group)

The total population is N = S + I + R + G, conserved since:
$$\frac{dN}{dt} = \frac{dS}{dt} + \frac{dI}{dt} + \frac{dR}{dt} + \frac{dG}{dt} = 0$$

\[
\text{As } t \rightarrow \infty, S \rightarrow 0, I \rightarrow 0, R \rightarrow N-G, G \rightarrow Ge^{-vt}
\]


The fraction getting GBS depends on the initial G and vaccination rate v. It decreases exponentially, but is never driven to zero if v  $< \infty$.

So in summary, while screening and vaccination could greatly reduce GBS incidence, eliminating the allele completely from the population is challenging, especially with a large starting frequency. Let me know if you have any other questions!


\end{document}
