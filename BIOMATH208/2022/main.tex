\documentclass{article}
\usepackage{mathtools}
\usepackage{graphicx} % Required for inserting images
\usepackage[a4paper, total={7in, 9in}]{geometry}
\usepackage{minted}
\usepackage{amsfonts} % Add this line to include the amsfonts package
\usepackage{datetime} % For date if required


\usepackage{algorithm}
\usepackage{algpseudocode} % Part of algorithmicx package

\usepackage{amsmath}
\DeclareMathOperator*{\argmin}{arg\,min}  % The asterisk is used to place the subscript under "arg min" in display style


\setlength\parindent{0pt}


\title{BIOMATH 208 Comp Exam Attempts}
\author{Simon A. Lee}
\date{}

\begin{document}

\maketitle

\section*{2022 Exam}
\section{Chapter 1}
\subsection{Discrete images and interpolation}

\textbf{Sample Values Calculation}

For \( i = 0, 1, 2, 3, 4 \), the values \( I_i \) are calculated as follows:
- \( I_0 = (0 - 2)^2 = 4 \)
- \( I_1 = (1 - 2)^2 = 1 \)
- \( I_2 = (2 - 2)^2 = 0 \)
- \( I_3 = (3 - 2)^2 = 1 \)
- \( I_4 = (4 - 2)^2 = 4 \)

\textbf{Pixel Locations}

Given the first pixel at \( O = -1.0 \) and \( \Delta = 0.5 \), the pixel locations are:
- \( x_0 = -1.0 \)
- \( x_1 = -0.5 \)
- \( x_2 = 0.0 \)
- \( x_3 = 0.5 \)
- \( x_4 = 1.0 \)

\textbf{Interpolation at \( x = 0.75 \)}

Since \( x = 0.75 \) falls between \( x_3 = 0.5 \) and \( x_4 = 1.0 \), we use linear interpolation:
- Formula: \( I(x) = I_3 + \frac{(x - x_3)(I_4 - I_3)}{x_4 - x_3} \)
- Plugging in values: \( I(0.75) = 1 + \frac{(0.75 - 0.5)(4 - 1)}{1.0 - 0.5} \)
- Calculation: \( I(0.75) = 1 + \frac{0.25 \times 3}{0.5} = 1 + 1.5 = 2.5 \)

\textbf{Plotting the Image}

\begin{figure}[h!]
    \centering
    \includegraphics[width=0.5\linewidth]{1.1.png}
    \caption{Plotting I(x)}
    \label{fig:enter-label}
\end{figure}

Here's the plot representing the discrete image \( I(x) \) along with linear interpolation. The discrete points are shown in blue, and the interpolated line is shown in gray. The red point indicates the interpolated value at \( x = 0.75 \), which is 2.5.

\subsection{Window and Level}

The problem involves adjusting the windowing of an image based on the brightness levels of different regions of a computed tomography (CT) scan of the human brain. Let's break down the task and perform the necessary calculations.

\textbf{Understanding Windowing}
Windowing in imaging involves setting a lower and upper bound for pixel intensity (brightness), with intensities outside this range being clipped to either the minimum or maximum value. This technique is used to enhance the visibility of certain features in medical scans by adjusting the grayscale mapping.

\textbf{Given Data}
\begin{itemize}
    \item \textbf{Skull}: Brightness = 0.9
    \item \textbf{Brain's cortex}: Brightness = 0.4
    \item \textbf{Brain's white matter}: Brightness = 0.3
    \item \textbf{Brain's ventricles}: Brightness = 0.1
\end{itemize}

\textbf{Windowing Adjustments}

1. \textbf{Lower window = 0, Upper window = 1}: 
   - No changes occur as all values are within the window range.

2. \textbf{Lower window = 0, Upper window = 0.2}: 
   - Values above 0.2 are clipped to 0.2.

3. \textbf{Lower window = 0.35, Upper window = 0.45}: 
   - Values below 0.35 are clipped to 0.
   - Values above 0.45 are clipped to 0.45.

\textbf{Mathematical Mapping for the Adjusted }

Windowing
Using the formula for intensity remapping:
\[ I_{\text{new}} = \frac{\text{max\_out} - \text{min\_out}}{\text{max\_in} - \text{min\_in}} \cdot (I - \text{min\_in}) + \text{min\_out} \]
where:
\begin{itemize}
    \item - \( I \) is the original intensity,
    \item - \( \text{min\_in} \) and \( \text{max\_in} \) are the original window settings,
    \item - \( \text{min\_out} \) and \( \text{max\_out} \) are the target display settings (usually 0 to 1).
\end{itemize}

1. \textbf{Window 0 to 0.2}: 
   - All regions with original brightness above 0.2 are clipped to the maximum of 1.0 (rendered as black in the pie chart).
   - This results in the skull and cortex appearing black due to clipping.

2. \textbf{Window 0.35 to 0.45}:
   - Regions below 0.35 are clipped to 0 (rendered as white in the pie chart).
   - Regions above 0.45 are also clipped to 1.0 (black).
   - This leads to the ventricles and white matter appearing as white due to their low original brightness values.

\newpage
\section{Chapter 2}
\subsection{Change of vector components}

Given:
\begin{itemize}
    \item Basis \( A \) with vectors \( A_0 = (1, 0) \) and \( A_1 = (1, 1) \)
    \item Basis \( B \) with vectors \( B_0 = (0, 1) \) and \( B_1 = (1, 1) \)
    \item Vector \( x \) with components in basis \( A \): \( x^0 = 1 \), \( x^1 = 1 \)
\end{itemize}

The vector \( x \) in standard coordinates can be calculated by:
\[ x = x^0 A_0 + x^1 A_1 = 1 \cdot (1, 0) + 1 \cdot (1, 1) = (1, 0) + (1, 1) = (2, 1) \]

To find the components of \( x \) in basis \( B \), we first need to express \( B_0 \) and \( B_1 \) in terms of \( A \) and then solve for \( x \) in terms of \( B \).

The change of basis matrix \( P \) from \( A \) to \( B \) is formed by expressing \( B_0 \) and \( B_1 \) in terms of \( A \). \( B_0 = 0 \cdot A_0 + 1 \cdot A_1 = (1, 1) \) and \( B_1 = 1 \cdot A_0 + 0 \cdot A_1 = (1, 0) \) are incorrect since \( B_0 = (0, 1) \) and \( B_1 = (1, 1) \) by definition.

First, solve for coefficients \( b_0 \) and \( b_1 \) such that:
\[ B_0 = b_{00} A_0 + b_{01} A_1 \]
\[ B_1 = b_{10} A_0 + b_{11} A_1 \]

For \( B_0 = (0, 1) \):
\[ (0, 1) = b_{00} (1, 0) + b_{01} (1, 1) \]
\[ 0 = b_{00} + b_{01} \]
\[ 1 = b_{01} \]
Thus, \( b_{00} = -1 \) and \( b_{01} = 1 \).

For \( B_1 = (1, 1) \):
\[ (1, 1) = b_{10} (1, 0) + b_{11} (1, 1) \]
\[ 1 = b_{10} + b_{11} \]
\[ 1 = b_{11} \]
Thus, \( b_{10} = 0 \) and \( b_{11} = 1 \).

The change of basis matrix \( P \) is:
\[ P = \begin{pmatrix} -1 & 0 \\ 1 & 1 \end{pmatrix} \]

Now, to find \( x \) in basis \( B \):
\[ [x]_B = P^{-1} [x]_A \]
\[ [x]_A = \begin{pmatrix} 2 \\ 1 \end{pmatrix} \]
\[ P^{-1} = \begin{pmatrix} -1 & 0 \\ 1 & 1 \end{pmatrix}^{-1} = \begin{pmatrix} -1 & 0 \\ -1 & 1 \end{pmatrix} \]
\[ [x]_B = \begin{pmatrix} -1 & 0 \\ -1 & 1 \end{pmatrix} \begin{pmatrix} 2 \\ 1 \end{pmatrix} = \begin{pmatrix} -2 \\ -1 \end{pmatrix} \]

\subsection{Change of Covector Components}

Given:
- Covector \( \mu \) in the dual basis to \( A \) has components \( \mu_0 = 1 \), \( \mu_1 = 1 \)

To find components in the dual basis to \( B \):
\[ \mu_B = \mu_A P^{-1} \]
\[ \mu_A = \begin{pmatrix} 1 & 1 \end{pmatrix} \]
\[ \mu_B = \begin{pmatrix} 1 & 1 \end{pmatrix} \begin{pmatrix} -1 & 0 \\ -1 & 1 \end{pmatrix} = \begin{pmatrix} -2 & 1 \end{pmatrix} \]

Thus, \( \mu_B \) has components \( \mu_{B0} = -2 \) and \( \mu_{B1} = 1 \).

\newpage
\section{Chapter 3}
\subsection{Distances Between Curves}

To compute the norm squared between the two curves, we need to:
\begin{enumerate}
    \item Identify the centers and tangents of each line segment
    \item Apply the Gaussian kernel to calculate the distances
    \item Sum up the results
\end{enumerate}

Let's start with the black curve (solid lines):

Points: $(0,0)$, $(1,0)$, $(2,1)$
Centers: $c_1 = (0.5, 0)$, $c_2 = (1.5, 0.5)$
Tangents: $t_1 = (1, 0)$, $t_2 = (1, 1)$

For the gray curve (broken lines):

Points: $(0,1)$, $(1,1)$, $(2,0)$
Centers: $c_1' = (0.5, 1)$, $c_2' = (1.5, 0.5)$
Tangents: $t_1' = (1, 0)$, $t_2' = (1, -1)$

Now, let's calculate the distances using the Gaussian kernel:

\begin{enumerate}
    \item Between centers:
    \begin{align*}
        d_1 &= \exp(-|(0.5,0) - (0.5,1)|^2 / 2) = \exp(-0.5) \approx 0.6065 \\
        d_2 &= \exp(-|(1.5,0.5) - (1.5,0.5)|^2 / 2) = \exp(0) = 1
    \end{align*}

    \item Between tangents:
    \begin{align*}
        d_3 &= \exp(-|(1,0) - (1,0)|^2 / 2) = \exp(0) = 1 \\
        d_4 &= \exp(-|(1,1) - (1,-1)|^2 / 2) = \exp(-2) \approx 0.1353
    \end{align*}
\end{enumerate}

The norm squared is the sum of these distances:
\[
\text{Norm}^2 = d_1 + d_2 + d_3 + d_4 \approx 0.6065 + 1 + 1 + 0.1353 \approx 2.7418
\]

\subsection{Orientation matters }

Now we need to reorder the points and recalculate:

Black curve (left to right): $(0,0)$, $(1,0)$, $(2,1)$
Gray curve (right to left): $(2,0)$, $(1,1)$, $(0,1)$

Black curve remains the same:
Centers: $c_1 = (0.5, 0)$, $c_2 = (1.5, 0.5)$
Tangents: $t_1 = (1, 0)$, $t_2 = (1, 1)$

Gray curve (reordered):
Centers: $c_1' = (1.5, 0.5)$, $c_2' = (0.5, 1)$
Tangents: $t_1' = (-1, 1)$, $t_2' = (-1, 0)$

Recalculating distances:

\begin{enumerate}
    \item Between centers:
    \begin{align*}
        d_1 &= \exp(-|(0.5,0) - (1.5,0.5)|^2 / 2) = \exp(-0.625) \approx 0.5353 \\
        d_2 &= \exp(-|(1.5,0.5) - (0.5,1)|^2 / 2) = \exp(-0.625) \approx 0.5353
    \end{align*}

    \item Between tangents:
    \begin{align*}
        d_3 &= \exp(-|(1,0) - (-1,1)|^2 / 2) = \exp(-2.5) \approx 0.0821 \\
        d_4 &= \exp(-|(1,1) - (-1,0)|^2 / 2) = \exp(-2.5) \approx 0.0821
    \end{align*}
\end{enumerate}

The new norm squared:
\[
\text{Norm}^2 = d_1 + d_2 + d_3 + d_4 \approx 0.5353 + 0.5353 + 0.0821 + 0.0821 \approx 1.2348
\]

As we can see, the orientation of the curves significantly affects the calculated distance between them.

\newpage
\section{Chapter 4}


\subsection{Chart transitions (10 points)}

\textbf{Computing $y \circ x^{-1}$}

Given:
\begin{align*}
x(p) &= (p^0, p^1) \\
y(p) &= (p^0d/p^2, p^1d/p^2)
\end{align*}

Let $(u, v) = x(p) = (p^0, p^1)$. Then $p^2 = \sqrt{r^2 - u^2 - v^2}$ (since $p$ is on a sphere of radius $r$).

Therefore:
\begin{equation*}
y \circ x^{-1}(u, v) = \left(\frac{ud}{\sqrt{r^2 - u^2 - v^2}}, \frac{vd}{\sqrt{r^2 - u^2 - v^2}}\right)
\end{equation*}

\textbf{Computing $x \circ y^{-1}$}

Let $(a, b) = y(p) = (p^0d/p^2, p^1d/p^2)$

From the hint, if $X$ is the Euclidean distance from the origin in chart $x$, and $Y$ in chart $y$, then:

\begin{equation*}
Y = X \cdot \frac{d}{\sqrt{r^2 - X^2}}
\end{equation*}

We can see that $Y^2 = a^2 + b^2$ and $X^2 = (p^0)^2 + (p^1)^2$

Solving for $X$:

\begin{align*}
Y^2 &= X^2 \cdot \frac{d^2}{r^2 - X^2} \\
Y^2(r^2 - X^2) &= X^2d^2 \\
Y^2r^2 - Y^2X^2 &= X^2d^2 \\
Y^2r^2 &= X^2(d^2 + Y^2) \\
X^2 &= \frac{Y^2r^2}{d^2 + Y^2} \\
X &= \frac{Yr}{\sqrt{d^2 + Y^2}}
\end{align*}

The scale factor from $y$ to $x$ is $\frac{X}{Y} = \frac{r}{\sqrt{d^2 + Y^2}}$

Therefore:
\begin{equation*}
x \circ y^{-1}(a, b) = \left(\frac{ar}{\sqrt{d^2 + a^2 + b^2}}, \frac{br}{\sqrt{d^2 + a^2 + b^2}}\right)
\end{equation*}

\textbf{Proving Smoothness Compatibility}

To show that these charts are smoothness compatible, we need to prove that both transition maps are smooth (infinitely differentiable).

1. For $y \circ x^{-1}$, the components are ratios of polynomials where the denominator is never zero (since $r^2 - u^2 - v^2 > 0$ for points on the hemisphere). Such functions are always smooth.

2. For $x \circ y^{-1}$, the components involve a square root in the denominator, but the expression under the square root is always positive ($d^2 + a^2 + b^2 > 0$). Therefore, this map is also smooth.

Since both transition maps are smooth, the atlas consisting of these two charts is smoothness compatible.

\newpage
\section{Chapter 5}

\subsection*{5.1 A shear group (5 points)}

To show that the family of matrices $G = \left\{\begin{pmatrix} 1 & \theta \\ 0 & 1 \end{pmatrix} : \theta \in \mathbb{R}\right\}$ forms a group, we need to verify the group axioms:

\begin{enumerate}
    \item \textbf{Closure}: For any $A, B \in G$, we have:
    \[
    \begin{pmatrix} 1 & \theta_1 \\ 0 & 1 \end{pmatrix} \begin{pmatrix} 1 & \theta_2 \\ 0 & 1 \end{pmatrix} = \begin{pmatrix} 1 & \theta_1 + \theta_2 \\ 0 & 1 \end{pmatrix} \in G
    \]
    
    \item \textbf{Associativity}: Matrix multiplication is associative.
    
    \item \textbf{Identity}: The identity element is $\begin{pmatrix} 1 & 0 \\ 0 & 1 \end{pmatrix} \in G$.
    
    \item \textbf{Inverse}: For any $\begin{pmatrix} 1 & \theta \\ 0 & 1 \end{pmatrix} \in G$, its inverse is $\begin{pmatrix} 1 & -\theta \\ 0 & 1 \end{pmatrix} \in G$.
\end{enumerate}

Therefore, $G$ forms a group under matrix multiplication.

\subsection*{5.2 Rotations by shears (5 points)}

To show that $H = \left\{\begin{pmatrix} 1 & \theta \\ \phi & 1 \end{pmatrix} : (\theta, \phi) \in \mathbb{R}^2\right\}$ is not a group, we can show that it's not closed under multiplication:

\[
\begin{pmatrix} 1 & \theta_1 \\ \phi_1 & 1 \end{pmatrix} \begin{pmatrix} 1 & \theta_2 \\ \phi_2 & 1 \end{pmatrix} = \begin{pmatrix} 1 + \theta_1\phi_2 & \theta_1 + \theta_2 \\ \phi_1 + \phi_2 & 1 + \phi_1\theta_2 \end{pmatrix}
\]

The resulting matrix is not in $H$ unless $\theta_1\phi_2 = \phi_1\theta_2 = 0$.

To show that the product forms a rotation matrix:

\[
\begin{pmatrix} 1 & \alpha \\ 0 & 1 \end{pmatrix} \begin{pmatrix} 1 & 0 \\ \beta & 1 \end{pmatrix} \begin{pmatrix} 1 & \alpha \\ 0 & 1 \end{pmatrix} = \begin{pmatrix} 1 + \alpha\beta & 2\alpha + \alpha^2\beta \\ \beta & 1 + \alpha\beta \end{pmatrix}
\]

For this to be a rotation matrix, we need:

\begin{align*}
1 + \alpha\beta &= \cos\theta \\
\beta &= \sin\theta \\
2\alpha + \alpha^2\beta &= \sin\theta
\end{align*}

From the given conditions:
\[
\alpha = -\tan(\theta/2) = -\frac{\sin\theta}{1 + \cos\theta} \quad \text{and} \quad \beta = \sin\theta
\]

Substituting these into the rotation matrix equations and using trigonometric identities, we can verify that they indeed form a rotation matrix.

The efficiency of shears in computer graphics comes from their simple implementation, requiring only addition operations to shift rows or columns.

\newpage
\section{Chapter 6}

\subsection*{6.1 Change of coordinates of tangent vectors (10 points)}

To find the velocity in chart $x$, we need to apply the chart-induced change of basis formula. This involves calculating the Jacobian of the transition map from chart $y$ to chart $x$ at the given point.

Let's approach this step-by-step:

\begin{enumerate}
    \item The point in chart $x$ is $(r/2, r/2)$.
    
    \item We need to find $y(x^{-1}(r/2, r/2))$. From the problem statement in question 4 (which we assume has been solved), we know that:
    
    $y(x^{-1}(a, b)) = (\frac{a^2-b^2}{a^2+b^2}, \frac{2ab}{a^2+b^2})$
    
    \item Substituting $a = b = r/2$:
    
    $y(x^{-1}(r/2, r/2)) = (0, \frac{r^2/2}{r^2/2}) = (0, 1)$
    
    \item Now, we need to calculate the Jacobian of the transition map $y \circ x^{-1}$ at $(r/2, r/2)$:
    
    $J_{y \circ x^{-1}}|_{(r/2, r/2)} = \begin{pmatrix}
    \frac{\partial (y_1 \circ x^{-1})}{\partial x_1} & \frac{\partial (y_1 \circ x^{-1})}{\partial x_2} \\
    \frac{\partial (y_2 \circ x^{-1})}{\partial x_1} & \frac{\partial (y_2 \circ x^{-1})}{\partial x_2}
    \end{pmatrix}_{(r/2, r/2)}$
    
    \item Calculating these partial derivatives and evaluating at $(r/2, r/2)$:
    
    $J_{y \circ x^{-1}}|_{(r/2, r/2)} = \begin{pmatrix}
    0 & -2/r \\
    2/r & 0
    \end{pmatrix}$
    
    \item The change of basis formula is:
    
    $v_x = (J_{y \circ x^{-1}}|_{(r/2, r/2)})^{-1} \cdot v_y$
    
    \item The inverse of the Jacobian is:
    
    $(J_{y \circ x^{-1}}|_{(r/2, r/2)})^{-1} = \begin{pmatrix}
    0 & r/2 \\
    -r/2 & 0
    \end{pmatrix}$
    
    \item Finally, we can calculate the velocity in chart $x$:
    
    $v_x = \begin{pmatrix}
    0 & r/2 \\
    -r/2 & 0
    \end{pmatrix} \begin{pmatrix}
    1 \\ -1
    \end{pmatrix} = \begin{pmatrix}
    -r/2 \\ -r/2
    \end{pmatrix}$
\end{enumerate}

Therefore, the surgeon should guide the robot with velocity $(-r/2, -r/2)$ in chart $x$ to follow the linear structure.

\newpage
\section{Chapter 7}
\subsection{An optimal solution}

Given two sets of points \( P \) and \( Q \) in \( \mathbb{R}^2 \), find shear parameters \( \theta \) and \( \phi \) such that:
\[
S = \begin{pmatrix} 1 & \theta \\ \phi & 1 \end{pmatrix}
\]
Minimize:
\[
E(\theta, \phi) = \sum_{i=1}^N \|SQ_i - P_i\|^2 = \sum_{i=1}^N \left[ (x_{Q_i} + \theta y_{Q_i} - x_{P_i})^2 + (\phi x_{Q_i} + y_{Q_i} - y_{P_i})^2 \right]
\]
Partial derivatives give:
\[
\frac{\partial E}{\partial \theta} = \sum_{i=1}^N 2(x_{Q_i} + \theta y_{Q_i} - x_{P_i})y_{Q_i} = 0
\]
\[
\frac{\partial E}{\partial \phi} = \sum_{i=1}^N 2(\phi x_{Q_i} + y_{Q_i} - y_{P_i})x_{Q_i} = 0
\]
Solutions:
\[
\theta = \frac{\sum_{i=1}^N (x_{P_i} - x_{Q_i})y_{Q_i}}{\sum_{i=1}^N y_{Q_i}^2}
\]
\[
\phi = \frac{\sum_{i=1}^N (y_{P_i} - y_{Q_i})x_{Q_i}}{\sum_{i=1}^N x_{Q_i}^2}
\]

\newpage
\section{Chapter 8}

\subsection{Derivate of the Metric}
Given the metric tensor
\[
g_{ij}(q) = \begin{pmatrix}
1 + \partial_0 f(q) \partial_0 f(q) & \partial_0 f(q) \partial_1 f(q) \\
\partial_0 f(q) \partial_1 f(q) & 1 + \partial_1 f(q) \partial_1 f(q)
\end{pmatrix},
\]
where \( \partial_0 \) and \( \partial_1 \) denote the partial derivatives with respect to the first and second argument respectively, we compute the partial derivatives of the metric elements \( g_{ij} \).


Given the metric tensor components as follows:
\[
g_{00} = 1 + \partial_0 f \partial_0 f, \quad g_{01} = \partial_0 f \partial_1 f, \quad g_{11} = 1 + \partial_1 f \partial_1 f,
\]
we calculate the partial derivatives with respect to coordinates \( q_0 \) and \( q_1 \) as:

\textbf{\( \partial_0 g_{00} \) and \( \partial_1 g_{00} \)}
\[
\partial_0 g_{00} = 2 \partial_0 \partial_0 f \cdot \partial_0 f
\]
\[
\partial_1 g_{00} = 2 \partial_1 \partial_0 f \cdot \partial_0 f
\]

\textbf{\( \partial_0 g_{01} \) and \( \partial_1 g_{01} \) (Symmetrically \( \partial_0 g_{10} \) and \( \partial_1 g_{10} \))}
\[
\partial_0 g_{01} = \partial_0 \partial_0 f \cdot \partial_1 f + \partial_0 f \cdot \partial_0 \partial_1 f
\]
\[
\partial_1 g_{01} = \partial_1 \partial_0 f \cdot \partial_1 f + \partial_0 f \cdot \partial_1 \partial_1 f
\]

\textbf{\( \partial_0 g_{11} \) and \( \partial_1 g_{11} \)}
\[
\partial_0 g_{11} = 2 \partial_0 \partial_1 f \cdot \partial_1 f
\]
\[
\partial_1 g_{11} = 2 \partial_1 \partial_1 f \cdot \partial_1 f
\]

\subsection{Christoffel Symbols of the First Kind}

Given the derivatives:
\[
\partial_0 g_{00} = 2 \partial_0 \partial_0 f \cdot \partial_0 f, \quad \partial_1 g_{00} = 2 \partial_1 \partial_0 f \cdot \partial_0 f, \quad \text{etc.}
\]
we calculate the Christoffel symbols using:
\[
\Gamma_{k,ij} = \frac{1}{2} \left( -\partial_k g_{ij} + \partial_i g_{kj} + \partial_j g_{ik} \right)
\]

\textbf{Example Calculation for \(\Gamma_{0,01}\)}
\[
\Gamma_{0,01} = \frac{1}{2} \left( -\partial_0 g_{01} + \partial_0 g_{00} + \partial_1 g_{00} \right)
\]
Substituting the derivatives:
\[
\Gamma_{0,01} = \frac{1}{2} \left( -(\partial_0 \partial_0 f \cdot \partial_1 f + \partial_0 f \cdot \partial_0 \partial_1 f) + 2 \partial_0 \partial_0 f \cdot \partial_0 f + 2 \partial_1 \partial_0 f \cdot \partial_0 f \right)
\]
\[
\Gamma_{0,01} = \partial_0 f \partial_0 \partial_1 f
\]

\textbf{Complete Calculation for All Symbols}
\[
\Gamma_{0,00} = \partial_0 f \partial_0 \partial_0 f, \quad \Gamma_{1,00} = \partial_0 f \partial_1 \partial_0 f
\]
\[
\Gamma_{0,01} = \partial_0 f \partial_0 \partial_1 f, \quad \Gamma_{1,01} = \partial_0 f \partial_1 \partial_1 f
\]
\[
\Gamma_{0,11} = \partial_1 f \partial_0 \partial_1 f, \quad \Gamma_{1,11} = \partial_1 f \partial_1 \partial_1 f
\]
\[
\Gamma_{0,10} = \partial_0 f \partial_0 \partial_1 f, \quad \Gamma_{1,10} = \partial_0 f \partial_1 \partial_1 f
\]

Thus, each Christoffel symbol of the first kind can be expressed as:
\[
\Gamma_{k,ij} = \partial_k f \partial_i \partial_j f
\]

\subsection{Christoffel Symbols of the Second Kind}

Given the inverse metric tensor:
\[
g^{-1} = \begin{pmatrix}
1 - \frac{(\partial_0 f)^2}{1 + |Df|^2} & -\frac{\partial_0 f \partial_1 f}{1 + |Df|^2} \\
-\frac{\partial_0 f \partial_1 f}{1 + |Df|^2} & 1 - \frac{(\partial_1 f)^2}{1 + |Df|^2}
\end{pmatrix}
\]
and the Christoffel symbols of the first kind:
\[
\Gamma_{k,ij} = \partial_k f \partial_i \partial_j f,
\]
the Christoffel symbols of the second kind are calculated as:
\[
\Gamma^l_{ij} = g^{lk} \Gamma_{k,ij} = g^{lk} \cdot (\partial_k f \partial_i \partial_j f)
\]

\textbf{Computation}
\[
\Gamma^l_{ij} = \sum_{k=0}^1 g^{lk} \Gamma_{k,ij} = \sum_{k=0}^1 g^{lk} \cdot (\partial_k f \partial_i \partial_j f)
\]
Simplifying, we assume summation over \(k\) and index manipulation:
\[
\Gamma^0_{ij} = g^{00} \Gamma_{0,ij} + g^{01} \Gamma_{1,ij} = \left(1 - \frac{(\partial_0 f)^2}{1 + |Df|^2}\right)(\partial_0 f \partial_i \partial_j f) - \frac{\partial_0 f \partial_1 f}{1 + |Df|^2} (\partial_1 f \partial_i \partial_j f)
\]
\[
\Gamma^1_{ij} = g^{10} \Gamma_{0,ij} + g^{11} \Gamma_{1,ij} = -\frac{\partial_0 f \partial_1 f}{1 + |Df|^2} (\partial_0 f \partial_i \partial_j f) + \left(1 - \frac{(\partial_1 f)^2}{1 + |Df|^2}\right)(\partial_1 f \partial_i \partial_j f)
\]

\textbf{Final Expression}
It is evident that the simplified form matches the expected structure:
\[
\Gamma^l_{ij} = \frac{1}{1 + |Df|^2} (\partial_l f \partial_i \partial_j f)
\]
This confirms the problem's statement that the Christoffel symbols of the second kind can be represented as:
\[
\Gamma^l_{ij} = \frac{1}{1 + |Df|^2} \partial_l f \partial_i \partial_j f
\]

\subsection{Geodesic Equations for Specific Functions}

The geodesic equation in a general form is given by:

\[
\ddot{q}^l + \Gamma^l_{ij}(q) \dot{q}^i \dot{q}^j = 0
\]

We'll explore this equation for two specific cases of the function \( f(u, v) \): when \( f \) is linear and when \( f(u, v) = -u^2 \).

\textbf{1. Linear Function \( f(u, v) \)}

Let \( f(u, v) = au + bv + c \) where \( a, b, \) and \( c \) are constants. The derivatives are:

\[
\partial_0 f = a, \quad \partial_1 f = b, \quad \partial_0 \partial_0 f = 0, \quad \partial_0 \partial_1 f = 0, \quad \partial_1 \partial_1 f = 0
\]

Substituting these into the Christoffel symbols of the second kind (from the previous discussion):

\[
\Gamma^l_{ij} = \frac{1}{1 + |Df|^2} \partial_l f \partial_i \partial_j f
\]

Since all second derivatives are zero, all Christoffel symbols \( \Gamma^l_{ij} \) vanish. Hence, the geodesic equation simplifies to:

\[
\ddot{q}^l = 0
\]

This equation indicates that the acceleration components \( \ddot{q}^l \) are zero, implying that the paths are straight lines in the \( u,v \)-plane.

\textbf{2. Function \( f(u, v) = -u^2 \)}

Here, \( f(u, v) = -u^2 \), so the partial derivatives are:

\[
\partial_0 f = -2u, \quad \partial_1 f = 0, \quad \partial_0 \partial_0 f = -2, \quad \partial_0 \partial_1 f = 0, \quad \partial_1 \partial_1 f = 0
\]

Again, using the formula for the Christoffel symbols of the second kind:

\[
\Gamma^l_{ij} = \frac{1}{1 + |Df|^2} \partial_l f \partial_i \partial_j f
\]

The only nonzero Christoffel symbol is \( \Gamma^0_{00} \):

\[
\Gamma^0_{00} = \frac{1}{1 + 4u^2}(-2u)(-2) = \frac{4u}{1 + 4u^2}
\]

Hence, the geodesic equation becomes:

\[
\ddot{u} + \frac{4u}{1 + 4u^2} \dot{u}^2 = 0, \quad \ddot{v} = 0
\]

The solution for \( v(t) \) indicates straight-line motion in the \( v \)-direction, while the solution for \( u(t) \) involves a more complex interplay due to the curvature of the surface.

\newpage
\section{Chapter 9}

\textbf{9.1 Procrustes Algorithm}

We start with two data points, \( s_1 = 1 \) and \( s_2 = 4 \), on a scale space with an initial guess \( s_0 = 1 \) (base point).

\textbf{Steps for Procrustes Algorithm:}

\textbf{1. Compute the Riemannian Logarithm:}
   \[
   \text{log}_1(4) = \log(4/1) = \log(4)
   \]

\textbf{2. Average the Velocity:}
   \[
   v = \text{Average}(\log(1), \log(4)) = \frac{0 + \log(4)}{2} = \frac{\log(4)}{2}
   \]

\textbf{3. Compute the Riemannian Exponential:}
   \[
   \text{exp}_1\left(\frac{\log(4)}{2}\right) = \exp\left(\frac{\log(4)}{2}\right) = 2
   \]

   Repeat the steps for the new average:

\textbf{4. Riemannian Logarithm Again:}
   \[
   \text{log}_2(4) = \log(4/2) = \log(2)
   \]

\textbf{5. Average the Velocity Again:}
   \[
   v = \frac{\log(2)}{2}
   \]

\textbf{6. Compute the Riemannian Exponential Again:}
   \[
   \text{exp}_2\left(\frac{\log(2)}{2}\right) = \exp\left(\frac{\log(2)}{2}\right) = \sqrt{4} = 2
   \]

The algorithm converges at \( s = 2 \).

\textbf{9.2 Interpolation Algorithm}

The interpolation algorithm is simply a single geodesic curve computation:

1. \textbf{Set up Geodesic:}
   \[
   q(t) = \exp\left(t \cdot \log\left(\frac{4}{1}\right)\right) = \exp(t \cdot \log(4))
   \]

2. \textbf{Evaluate at \( t = 0.5 \):}
   \[
   q(0.5) = \exp(0.5 \cdot \log(4)) = 2
   \]

Both methods converge to the same result, \( s = 2 \), showing that the Riemannian average of the scales \( 1 \) and \( 4 \) is \( 2 \).

\end{document}

