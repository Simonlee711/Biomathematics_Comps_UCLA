\documentclass{article}
\usepackage{graphicx} % Required usepackage{mathtools}
\usepackage{graphicx} % Required for inserting images
\usepackage[a4paper, total={7in, 9in}]{geometry}
\usepackage{minted}
\usepackage{amsfonts} % Add this line to include the amsfonts package
\usepackage{datetime} % For date if required


\usepackage{algorithm}
\usepackage{algpseudocode} % Part of algorithmicx package

\usepackage{amsmath}
\DeclareMathOperator*{\argmin}{arg\,min}  % The asterisk is used to place the subscript under "arg min" in display style


\setlength\parindent{0pt}

\title{Biomath 202 2012 Exam}
\author{SIMON LEE}
\date{July 2024}

\begin{document}

\maketitle

\section{Answers}

\subsection{Question 1}

Let's start with the equation for fitness of a population exposed to both perturbations:

\[w_{xy} = w_0 + a_x x + a_y y + \epsilon_{xy} xy\]

Where:
- $w_0$ is the baseline fitness
- $a_x$ and $a_y$ are the individual effects of perturbations x and y
- $\epsilon_{xy}$ is the interaction term (analogous to covariance)

To solve for $w_{xy}$, we consider the individual fitnesses:

\[w_x = w_0 + a_x x\]
\[w_y = w_0 + a_y y\]

Now we can express $w_{xy}$ in terms of these:

\[w_{xy} = w_x + w_y - w_0 + \epsilon_{xy} xy\]

This is the equation that allows us to classify interactions as additive, synergistic, or antagonistic:

- If $\epsilon_{xy} = 0$, the interaction is additive (no interaction)
- If $\epsilon_{xy} > 0$, the interaction is synergistic
- If $\epsilon_{xy} < 0$, the interaction is antagonistic

\subsection{Question 2}

Along the lines x=0 and y=0, the solution from problem 1 reduces to:

For x=0: 
\[w_{xy} = w_y = w_0 + a_y y\]

For y=0: 
\[w_{xy} = w_x = w_0 + a_x x\]

This means that when one perturbation is absent (x=0 or y=0), the fitness is only affected by the other perturbation. The interaction term $\epsilon_{xy} xy$ becomes zero, and we're left with the linear effect of the remaining perturbation.

\subsection{Question 3}

For small changes dx and dy, we can express the total change in fitness as:

\[dw_{xy} = \frac{\partial w_{xy}}{\partial x} dx + \frac{\partial w_{xy}}{\partial y} dy\]

To find the partial derivatives:

\[\frac{\partial w_{xy}}{\partial x} = a_x + \epsilon_{xy} y\]
\[\frac{\partial w_{xy}}{\partial y} = a_y + \epsilon_{xy} x\]

Substituting these back into the equation:

\[dw_{xy} = (a_x + \epsilon_{xy} y) dx + (a_y + \epsilon_{xy} x) dy\]

We can rewrite this in terms of the individual fitnesses and the covariance:

\[dw_{xy} = \frac{\partial w_x}{\partial x} dx + \frac{\partial w_y}{\partial y} dy + \epsilon_{xy} (y dx + x dy)\]

This expression includes:
- The sum of partial derivatives of the individual fitnesses: $\frac{\partial w_x}{\partial x} dx + \frac{\partial w_y}{\partial y} dy$
- The covariance term: $\epsilon_{xy} (y dx + x dy)$

This formulation allows us to see how small changes in the perturbation values affect the overall fitness, taking into account both the individual effects and their interaction.

\subsection{Question 4}

Given: Cov$(w_x,w_y) = (x-cy)^p$, where c is a constant coefficient and p is an exponent.

To derive an expression for $\frac{\partial w_{xy}}{\partial x}$ when there's a change dx in x but no change in y:

1) Recall from the previous answer that:

   $$\frac{\partial w_{xy}}{\partial x} = a_x + \epsilon_{xy} y$$

2) Here, $\epsilon_{xy}$ is related to the covariance. We can express it as:

   $$\epsilon_{xy} = \frac{\partial^2 w_{xy}}{\partial x \partial y} = \frac{\partial}{\partial x}(x-cy)^p$$

3) Applying the chain rule:

   $$\epsilon_{xy} = p(x-cy)^{p-1} \cdot \frac{\partial}{\partial x}(x-cy) = p(x-cy)^{p-1}$$

4) Therefore, the expression for $\frac{\partial w_{xy}}{\partial x}$ is:

   $$\frac{\partial w_{xy}}{\partial x} = a_x + p(x-cy)^{p-1}y$$

Now, let's consider when the covariance term does not contribute:

The covariance term doesn't contribute when $p(x-cy)^{p-1}y = 0$. This occurs when:

a) $y = 0$, regardless of the values of x, c, or p.
b) $x = cy$, making $(x-cy)^{p-1} = 0^{p-1}$, which is undefined for $p < 1$, zero for $p > 1$, and 1 for $p = 1$ (but in this case, $y$ must be zero for no contribution).

This does depend on the value of p:
\begin{itemize}
    \item - For $p = 1$, the covariance term is constant (py) and only doesn't contribute when $y = 0$.
    \item - For $p \neq 1$, the covariance term can be zero when either $y = 0$ or $x = cy$.
\end{itemize}

For $p = 1$, to eliminate the contribution of the covariance term with a change dy:

1) The covariance term is $$y(x-cy)^0 = y$$

2) To eliminate this, we need:

   $$y \cdot dx + x \cdot dy = 0$$

3) Solving for dy:

   $$dy = -\frac{y}{x} \cdot dx$$

This means that for a given change dx, you could eliminate the covariance term contribution by simultaneously changing y by $-\frac{y}{x} \cdot dx$. This creates a proportional change in y that exactly cancels out the covariance effect of the change in x.

\subsection{Question 5}

Given: Fitness levels monotonically decrease with perturbations x and y, and fitness cannot be negative.

For a change dx (without change in y):

1) Additivity: $$\frac{\partial Cov(w_x,w_y)}{\partial x} = 0, \frac{\partial w_{xy}}{\partial x} < 0$$

2) Synergy: $$\frac{\partial Cov(w_x,w_y)}{\partial x} < 0,\frac{\partial w_{xy}}{\partial x} < 0$$

3) Antagonism: $$\frac{\partial Cov(w_x,w_y)}{\partial x} > 0\frac{\partial w_{xy}}{\partial x} < 0$$

For a change dy (without change in x):

1) Additivity: $$\frac{\partial Cov(w_x,w_y)}{\partial y} = 0, \frac{\partial w_{xy}}{\partial y} < 0$$

2) Synergy: $$\frac{\partial Cov(w_x,w_y)}{\partial y} < 0, \frac{\partial w_{xy}}{\partial y} < 0$$

3) Antagonism: $$\frac{\partial Cov(w_x,w_y)}{\partial y} > 0, \frac{\partial w_{xy}}{\partial y} < 0$$

\subsection{Question 6}

For additivity, the equation from part 3 reduces to:

$$dw_{xy} = \frac{\partial w_x}{\partial x} dx + \frac{\partial w_y}{\partial y} dy$$

To remain on a line of constant fitness:

$$dw_{xy} = 0 = \frac{\partial w_x}{\partial x} dx + \frac{\partial w_y}{\partial y} dy$$

Solving for dy:

$$dy = -\frac{\frac{\partial w_x}{\partial x}}{\frac{\partial w_y}{\partial y}} dx$$

This shows that if dx increases, dy needs to decrease to maintain constant fitness. The exact relationship is:

$$dy = -\frac{w_x}{w_y} dx$$

Where $w_x$ and $w_y$ are the fitnesses for the single perturbations.

\subsection{Question 7}

For synergy and antagonism, we use the full equation from part 3:

$$dw_{xy} = \frac{\partial w_x}{\partial x} dx + \frac{\partial w_y}{\partial y} dy + \epsilon_{xy} (y dx + x dy)$$

To remain on lines of constant fitness, $$dw_{xy} = 0$$:

$$0 = \frac{\partial w_x}{\partial x} dx + \frac{\partial w_y}{\partial y} dy + \epsilon_{xy} (y dx + x dy)$$

Solving for dy:

$$dy = -\frac{\frac{\partial w_x}{\partial x} + \epsilon_{xy}y}{\frac{\partial w_y}{\partial y} + \epsilon_{xy}x} dx$$

For synergy ($\epsilon_{xy} < 0$):
\begin{itemize}
    \item - If dx increases, dy must decrease more than in the additive case.
    \item - If dx decreases, dy must increase more than in the additive case.
\end{itemize}

For antagonism ($\epsilon_{xy} > 0$):
\begin{itemize}
    \item - If dx increases, dy must decrease less than in the additive case.
    \item - If dx decreases, dy must increase less than in the additive case.
\end{itemize}

These results align with the trajectories in the figure: synergistic effects curve towards the origin, while antagonistic effects curve away from it.

\subsection{Question 8}

a) The covariance equation for the effect of genes X and Y:

$$Cov(X,Y) = E_{xy} - E_x - E_y + E_0$$

b) Additive effects would lead to $$Cov(X,Y) = 0$$:

$$E_{xy} = E_x + E_y - E_0$$

To modify for non-zero results, introduce an interaction term $\epsilon$:

$$E_{xy} = E_x + E_y - E_0 + \epsilon$$

c) Non-zero covariance interpretation:
\begin{itemize}
    \item - Positive covariance: synergistic effect (proteins enhance each other's effect)
    \item - Negative covariance: antagonistic effect (proteins inhibit each other's effect)
\end{itemize}

The analysis from parts 1-7 can apply to this system:
\begin{itemize}
    \item - X and Y correspond to perturbations
    \item - Expression levels correspond to fitness
    \item - The covariance term represents the interaction between X and Y proteins
\end{itemize}

This framework allows us to analyze gene interactions similarly to how we analyzed perturbation interactions, providing insights into the regulatory dynamics of gene Z.

I'll answer these questions, showing all relevant math steps.

\subsection{Question 9}

a) For random Erdős-Rényi graphs, the scaling of network motifs with N nodes:

$$M \approx \binom{N}{n} p^g$$

Where:
- M is the number of motifs
- n is the number of nodes in the motif
- g is the number of edges in the motif
- p is the probability of an edge existing between any two nodes

Explanation: We choose n nodes from N in $\binom{N}{n}$ ways, and the probability of g specific edges existing is $p^g$.

b) For geometric Itzkowitz-Alon graphs, the scaling is:

$$M \approx N \cdot (\rho R^d)^{n-1}$$

Where:
- $\rho$ is the density of nodes
- $R$ is the radius of influence
- $d$ is the dimensionality of the space

(c) For Itzkowitz-Alon graphs, the expected number of wedge motifs \[ \langle G_{\wedge} \rangle \] with a Gaussian connectivity function:

\[
\langle G_{\wedge} \rangle \approx N \cdot (2\pi R^2)^2 \cdot \exp\left(-\frac{r^2}{2R^2}\right)
\]

Where \( r \) is the distance between nodes.

\subsection{Question 9.a}

The variance in the number of wedge motifs, \[\text{Var}(G_{\wedge})\], can be calculated as:

\[
\text{Var}(G_{\wedge}) = \langle G_{\wedge}^2 \rangle - \langle G_{\wedge} \rangle^2
\]

Where:
\[
\langle G_{\wedge}^2 \rangle = \sum_{i,j,k} \sum_{l,m,n} P(i,j,k \text{ form a wedge}) \cdot P(l,m,n \text{ form a wedge})
\]
\[
\langle G_{\wedge} \rangle = \sum_{i,j,k} P(i,j,k \text{ form a wedge})
\]

To compute numerically:
\begin{enumerate}
\item Generate many random graphs
\item Count wedges in each graph
\item Calculate mean and variance of wedge counts
\end{enumerate}

\subsection{Question 9.b}

Given that \( \langle G_{\wedge} \rangle \) scales with \( N \), we can deduce:

\[
\langle G_{\wedge}^2 \rangle \geq \langle G_{\wedge} \rangle^2 \sim N^2
\]

Therefore, \(\text{Var}(G_{\wedge}) \geq 0\) and scales at least as \( N^2 \).

Upper bound: \(\text{Var}(G_{\wedge}) \leq \langle G_{\wedge}^2 \rangle \sim N^3\) (if all possible wedges were independent)

\subsection{Question 9.c}

For geometric graphs with connectivity \( \langle k \rangle \) and radius \( R \):

In the limit of large \( \langle k \rangle \): 
\[
\text{Var}(G_{\wedge}) \sim N \cdot \langle k \rangle^2
\]

In the limit of small \( R \):
\[
\text{Var}(G_{\wedge}) \sim N \cdot R^{2d}
\]

Where \( d \) is the dimensionality of the space.

\subsection{Question 10.a}

The variance relates to the covariance of two wedge motifs as:

\[
\text{Var}(G_{\wedge}) = \text{Cov}(G_{\wedge}, G_{\wedge})
\]

For the covariance of a wedge with a triangle motif:

\[
\text{Cov}(G_{\wedge}, G_{\Delta}) = \langle G_{\wedge} G_{\Delta} \rangle - \langle G_{\wedge} \rangle \langle G_{\Delta} \rangle
\]

Where:
\[
\langle G_{\wedge} G_{\Delta} \rangle = \sum_{i,j,k} \sum_{l,m,n} P(i,j,k \text{ form a wedge and } l,m,n \text{ form a triangle})
\]

\subsection{Question 10.b}

For Erdős-Rényi graphs, \(\text{Cov}(G_{\wedge}, G_{\Delta})\) is expected to be non-zero and positive.

Reasoning: The presence of a wedge increases the likelihood of a triangle in the same local area, as they share structural elements.

\subsection{Question 10.c}

For Itzkowitz-Alon geometric graphs, \(\text{Cov}(G_{\wedge}, G_{\Delta})\) is expected to be non-zero and positive.

Reasoning: Spatial proximity that favors wedge formation also increases triangle likelihood.

As \( R \) increases:
\begin{itemize}
\item \( \langle G_{\wedge} \rangle \) increases (more potential connections)
\item \( \langle G_{\Delta} \rangle \) increases (more potential triangles)
\end{itemize}

This would likely increase the covariance, maintaining a positive sign.

The value of the covariance would increase with \( R \), as more motifs of both types would form, increasing their correlation. The sign would remain positive due to the inherent spatial relationship between wedges and triangles in geometric graphs.


\end{document}
